Determining the mechanisms by which genetic variants drive molecular, cellular, and physiological phenotypes has proved to be challenging \citep{Schadt2009}. Genome-wide experiments that provide data on variations in molecular and cellular states in genotyped individuals can contribute to elucidating these mechanisms, but most analyses of these data are largely observational, due in part to constraints of specific populations (\textit{i.e.} humans), the ongoing development of experimental technologies, and the challenge of coordinating large numbers of experiments with multiple levels of data \citep{Schaid2018}. More recently, complementary genome-wide datasets from the same individuals have been paired with modern statistical mediation analysis to draw inferences on the relationships that connect these data. Results of these analyses will more likely identify causal rather than correlational interactions, providing meaningful and actionable targets in terms of downstream applications in areas such as medicine and agriculture.

In particular, integrative analyses can inform upon the regulation of fundamental biological processes, through the combined assessment of gene transcription and gene regulation data. \cite{Degner2012} correlated chromatin accessibility quantitative trait loci (cQTL), determined using DNase-seq, with gene expression QTL (eQTL) in human lymphoblastoid cell lines, to suggest how some genetic variants modulated transcription though chromatin state. \cite{Battle2014} investigated the regulation of gene expression in 922 humans, using eQTL and allele specific expression QTL to assess evidence for genes proximal to distal-eQTL behaving as mediators to distal-eQTL target genes. \cite{Pai2015} localized eQTL to genomic regulatory elements in human lymphoblastoid cell line data to infer the mode of action of the QTL. \cite{Alasoo2017} further elucidated the roles of eQTL within regulatory elements by testing the effect of enhancer sequence modifications. \cite{Roytman2018} propose the use of probabilistic mediation models, finding they perform better to detect relationships between histone modification QTL (hQTL) and eQTL in comparison to simple QTL overlap in data from 112 humans. \cite{Wu2018} use mediation to tease apart the relationships among DNA methylation sites, gene expression, and complex traits. \cite{Battle2015} detected the co-localization of eQTL, ribosome occupancy QTL (rQTL), and protein abundance (pQTL), observing significant overlap and a potential buffering of rQTL effects on protein levels.

Despite these successes, using data from human populations for genotype-phenotype analyses are challenging, in part due to an inability to tightly control experimental conditions and genetic backgrounds to reproducibly and unambiguously detect casual linkages. To aid in this effort, two genetically-diverse mouse population resources have been established, the Collaborative Cross (CC) \citep{Churchill2004,Hall2012,Srivastava2017} and the Diversity Outbred (DO) mice \citep{Churchill2012}. Derived from the same eight founder strains, the CC are recombinant inbred strains while the DO are largely heterogeneous outbred animals, bred with a random mating strategy that seeks to maximize diversity. \cite{Chick2016} used a genome-wide mediation approach to link transcriptional and post-translational regulation of protein levels in 192 DO mice, and also used the CC to confirm results by showing that estimates of founder allele effects from each of the related populations corresponded. In comparison to the DO, the CC is more restricted in the number of unique genomes and reduced mapping resolution, but it uniquely provides the ability for replicate observations under varying experimental conditions while maintaining a constant genetic background.

Here we use a sample composed of a single male mouse from 47 strains of the CC to investigate tissue-specific dynamics between chromatin accessibility, as determined by Assay for Transposase Accessible Chromatin sequencing (ATAC-seq), and gene expression in lung, liver, and kidney tissues. We detect QTL underlying gene expression and chromatin accessibility variation across the strains, and assess the support for mediation of the effect of eQTL on gene expression through chromatin accessibility, using a novel implementation of methods adapted from \cite{Chick2016}. Additionally, we detect gene mediators of distal-eQTL, as in \cite{Keller2018}. Concurrently with our work, others have used mediation analysis to connect chromatin accessibility with gene expression in the embryonic stem cells derived from the DO \citep{Skelly2019}. We identify and characterize examples of strong mediation, as well as co-localizing but independent eQTL and cQTL. These findings demonstrate the experimental power of the CC for integrative analysis of multi-omic data to determine genetically-driven phenotype variation, despite limited sample size, and provides support for continued use of the CC in larger experiments going forward.
