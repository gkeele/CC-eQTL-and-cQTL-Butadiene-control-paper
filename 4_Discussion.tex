\section{Discussion}

We performed QTL mapping on gene expression and chromatin accessibility data in 47 CC strains. We have described the use of multiple procedures of varying statistical stringency to allow for detection of lower effect local-QTL, important in a study with a restricted sample size. 
Furthermore, we described statistical approaches for two models of the mediation of the eQTL effects, one through chromatin accessibility for local-eQTL, and the other through activity of proximal genes as measured through their expression for distal-eQTL. We oriented our mediation through chromatin accessibility on local-eQTL, for which this study is better powered, and find significant support that mediation through chromatin is present (4.2-9.6\% of local-eQTL across the tissues). Certain regions of the genome could not be assessed for cQTL or chromatin mediation due to ATAC-seq read alignment ambiguity, further restricting our ability to assess mediation. Despite these constraints, our results are consistent with the role of accessible chromatin regions in the regulation of gene expression \citep{Klemm2019}. Despite being poorly powered to detect gene expression mediators of distal-eQTL we do find a number of strong candidates, including zinc finger protein genes. 

\subsection{Multiple procedures for QTL detection}

The use of multiple procedures for QTL detection allowed us to assess varying degrees of statistical support for their associations, and of particular importance, leverage prior belief in the predominance of local associations in order to detect higher numbers of QTL for a sample of 47 mice. The multi-stage procedure (Method 1) allowed for genome-wide detection of multiple QTL per outcome while incorporating both FWER and FDR adjustments. Both the chromosome-wide FDR (Method 2) and local-only (Method 3) analyses strongly increase power to detect local-QTL, but lose the ability to identify distal-QTL located on other chromosomes.

% \subsection{Distal-QTL on local chromosome}

% We find that we are powered to detect distal-QTL predominately located on the local chromosome, as seen in \textbf{Figure \ref{fig:grid_plot}}. Many FWER-significant (permP < 0.05) distal-QTL on non-local chromosomes are filtered by the FDR procedure, such as the distal-eQTL for \textit{Ccnyl1}, shown in \textbf{Figure \ref{fig:exmediation}}. The mediation of this distal-eQTL through zinc finger proteins strongly suggests that it is a real distal-eQTL. It is possible that distal-QTL on non-local chromosomes tend to be of lower effect than distal-QTL located on the local chromosome, though far from the gene or chromatin site, and these data are not sufficiently powered to detect non-local chromosome distal-QTL. 

\subsection{QTL effect size}

Estimation of the variance explained by the QTL, here referred to as the effect size, can provide a useful and interpretable point summary for haplotype-based QTL mapping, analogous to the effect of the minor allele, which is commonly used in SNP association. We used two estimates of effect size, one based on a fixed effect fitting of the QTL term (Eq \ref{eq:effect_size}) whereas the other included the QTL as a random term (Eq \ref{eq:effect_size_ranef}). Both estimates of effect size possessed unique and appealing qualities. We primarily report results from the fixed effects estimate because these were largely consistent with the expectations prescribed by \cite{KeeleSPARCC} for a study of this size, whereas the conservative shrinkage-based estimate seemed overly reduced, likely due to the sample size. The mapping results are largely consistent with the genetic regulation of these molecular traits being largely Mendelian, with large effect sizes (> 60 \%), though more complex genetic regulation is suggested by QTL with smaller effect sizes (< 50\%) when detecting local-QTL with reduced stringency. Nevertheless, these estimates emphasize that the study is not well-powered to detect QTL with effect sizes < 50\%.

The shrunken effect size estimates can potentially identify putative QTL with unstable fixed effects from influential data points. Notably, a number of the distal-eQTL have very low random effect size estimates (\textbf{Figure \ref{fig:qtl_effect_size_fixefvsranef}}) compared to their fixed effects-based estimates. Notably, QTL with this pattern of effect size estimates are largely distal. A mapping procedure that fits the QTL effect as a random effect, as in \cite{Wei2016}, would likely not detect these QTL. However, a random effects procedure is more computationally intensive, which is particularly unappealing in the context of genome-wide molecular traits, thus justifying a follow-up approach in which effect sizes are evaluated for the detected QTL.

\subsection{Joint QTL analysis in multiple tissues}

Recent research, such as the GTEx project \citep{GTEX2017}, comprised of expression data from 44 human tissues, represents a growing interest in interrogating and characterizing differences in gene expression and its regulation across a wide range of tissues. Previous studies in mice (\eg \citealt{Huang2009}) have used tissue-specific or stratified QTL analyses, similar to as done herein, and then characterized the overlap across tissues. Our use of formal statistical tests of the correlation coefficient between the founder allele effects of overlapping QTL is, to our knowledge, a novel approach to formally defining overlapping QTL, allowing for the identification of proximal QTL with unique founder effects patterns, likely representing distinct tissue-specific QTL, such as observed with three tissue-specific local-eQTL for the gene \textit{Pik3c2g} (\textbf{Figure \ref{fig:pik3c2g}}). 

However, examples like the gene \textit{Per2} (\textbf{Figure \ref{fig:correlated_distal_eqtl}B}), which possessed marginally significant intra-chromosomal distal-eQTL with consistent founder allele effects observed across the three tissues, suggest that mapping QTL in all the tissues jointly could better leverage the full extent of the data and detect QTL more powerfully. Formal joint analysis approaches have been proposed, largely implemented for SNP association, including meta-analysis on summary statistics (\eg \citealt{Fu2012a, Sul2013}) and fully joint analysis, including Bayesian hierarchical models \citep{Flutre2013} and mixed models \citep{Acharya2016}. Extending such statistical frameworks to haplotype-based analysis in MPP, as used here, poses some challenges, including how to best generalize these approaches to the more complex genetic model, and in the context of the CC where the number of unique genomes is limited. Nevertheless, such approaches would certainly be more powerful and principled for detecting and calling QTL present across multiple tissues, and should be considered for future related research in the CC and similar populations, such as the DO. When multiple levels of molecular traits are measured on each individual, as is the case for this study, there is also the potential to incorporate joint analysis approaches for QTL mapping into the mediation analysis, which could presumably improve power.

\subsection{Correlated QTL effects}

Haplotype-based association in MPP allows for the detection of multi-allelic QTL \citep{Aylor2011}, such as observed with the kidney local-eQTL of \textit{Pik3c2g} in which mice with B6 contributions in the region have an intermediate level of expression. The correlations between the founder allele effects for QTL pairs provides an interesting summary, generally not possible in the simpler bi-allelic setting commonly used in variant association. The extent of the correlation between the founder effects for QTL pairs of certain genes, such as \textit{Cox7c} and \textit{Ubc} (\textbf{Figure \ref{fig:correlated_local_eqtl}A}) seems to strongly support that these QTL are, at least subtly mutli-allelic, which suggests that genetic regulation, even local to the gene TSS, can be complex with strain-specific modifiers.

We detected numerous genes with QTL pairs with correlated founder effects while the overall expression patterns were significantly different based on the differential analysis, including \textit{Cox7c}, \textit{Ubc}, \textit{Slc44a3}, and \textit{Akr1e1}. We consider two main explanations for the unique co-occurence of consistent genetic regulation of genes with expression profiles that vary significantly across tissues. The samples used in the study represent bulk tissue samples, and thus possess cellular heterogeneity that likely varies between liver, lung, and kidney tissues. DE and DAR signal may simply represent the varying compositions of the tissues. Single cell sequence experiments would reduce noise from tissue heterogeneity, though the results would no longer reflect averages across the overall tissue. Statistical approaches that aim to estimate the cellular composition of a tissue \citep{Aran2017} have been developed, which could provide some insights into how the tissues vary from each other. Alternatively, DE genes that share QTL across tissues could be real, and suggest that other tissue-specific modifiers of expression are active that alter the magnitude of the expression profile. Single cell sequencing experiments could be used to confirm such patterns in expression.

\subsection{Mediation analysis}

Our approach to mediation analysis was similar to previous studies in DO mice \citep{Chick2016,Keller2018,Skelly2019}, though with some modifications, including the use of QTL effect sizes to establish consistency with the directionality of the mediation relationships and the formal calculation of a mediation p-value through permutation (permP).  To our knowledge, mediation analyses have not been used previously in the CC. Mediation within the field of system genetics is a growing area of interest, and the methods proposed here support that development.

Proving causality, particularly in a biological system as complex as transcriptional regulation in bulk tissue samples, is impossible. These mediation analyses will largely reflect the correlations between the variables after adjusting for additional sources of variation, such as covariates and batch effects. Our additional step of requiring the mediator QTL to have a larger effect size than outcome QTL aims to identify trios that are consistent with the proposed mediation models. It is possible that the measurement of the mediator is a noisier process compared to the measurement of the outcome, which would result in the effect size of the mediator QTL being reduced, and thus potentially a false negative mediation result. Complex relationships with many factors or levels of feedback are also unlikely to be detected.

Despite these limitations, mediation analysis can provide specific candidates underlying detected QTL, particularly in comparison to considering all genes near the QTL peak. The genetic regulation of \textit{Akr1e1} (\textbf{Figure \ref{fig:akr1e1_full_model}}) provides a strong example case in which a single gene, \textit{Zfp985}, in the eQTL region is identified as a strong mediator. It is possible that \textit{Zfp985} is not the true mediator, but is instead strongly correlated with it. For comparison, \cite{HamiltonWilliams2013} proposed \textit{Rex2} as a candidate, which was filtered out for having low expression from all three tissues in this study, whereas \textit{Zfp985} was lowly expressed but successfully quantified in lung. Disentangling the relationships underlying QTL poses a strong challenge, and mediation analyses can provide strong candidates for downstream experiments for validation. Additionally, mediation analysis provides strong support that the causal mediator inhibits \textit{Akr1e1} expression by reducing chromatin accessibility near its TSS.

\subsection{Conclusion}

In summary, our study has used QTL and mediation analyses of gene expression and chromatin accessibility data in 47 strains of the CC 









