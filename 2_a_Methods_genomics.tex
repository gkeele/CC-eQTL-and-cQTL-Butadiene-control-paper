\subsection{Animals}

Adult male mice (8-12 weeks old) from 47 CC strains (University of North Carolina Systems Genetics Core) were maintained on an NTP 2000 wafer diet (Zeigler Brothers, Inc., Gardners, PA) and water \textit{ad libitum}. The housing room was maintained on a 12-h light-dark cycle. Our experimental design sought to maximize the number of strains relative to within-strain replications based on the power analysis for QTL mapping in mouse populations \citep{Kaeppler1997}; therefore, one mouse was used per strain. Mice were euthanized and lungs, liver and kidney tissues were collected, flash frozen in liquid nitrogen, and stored at -80\degree C. These studies were approved by the Institutional Animal Care and Use Committees at Texas A\&M University and the University of North Carolina.

\subsection{mRNA sequencing and processing}

Total RNA was isolated from flash-frozen tissue samples using a Qiagen miRNeasy Kit (Valencia, CA) according to the manufacturer’s protocol. RNA purity and integrity were evaluated using a Thermo Scientific Nanodrop 2000 (Waltham, MA) and an Agilent 2100 Bioanalyzer (Santa Clara, CA), respectively. A minimum RNA integrity value of 7.0 was required for RNA samples to be used for library preparation and sequencing. Libraries for samples with a sufficient RNA integrity value were prepared using the Illumina TruSeq Total RNA Sample Prep Kit (Illumina, Inc., San Diego, USA) with ribosomal depletion. Single-end (50 bp) sequencing was performed (Illumina HiSeq 2500).

Sequencing reads were filtered (sequence quality score $\ge$ 20 for $\ge$ 90\% of bases) and adapter contamination was removed (TagDust). Reads were mapped to strain-specific pseudo-genomes (Build37, \url{http://csbio.unc.edu/CCstatus/index.py?run=Pseudo}) and psuedo-transcriptomes (C57BL/6J RefSeq annotations mapped to pseudo-genomes) using the RSEM with STAR (v2.5.3a). Uniquely aligned reads  were used to quantify expression as transcripts per million (TPM) values.

\subsection{ATAC-seq processing}

Flash frozen tissue samples were pulverized in liquid nitrogen using the BioPulverizer (Biospec) to break open cells and allow even exposure of intact chromatin to Tn5 transposase \citep{Buenrostro2015}. Pulverized material was thawed in glycerol containing nuclear isolation buffer to stabilize nuclear structure and then filtered through Miracloth (Calbiochem) to remove large tissue debris. Nuclei were washed and directly used for treatment with Tn5 transposase. Paired-end (50 bp) sequencing was performed (Illumina HiSeq 2500).

Reads were filtered as the mRNA were. Reads were aligned to the appropriate pseudo-genome using GSNAP (parameter set: -k 15, -m 1, -i 5, –sampling=1, –trim-mismatch-score=0, –genome-unk-mismatch=1, –query-unk-mismatch=1). Uniquely mapped reads were mapped to the mm9 mouse reference genome using the associated MOD files (UNC) to allow comparison across strains. Reads overlapping regions in the mm9 blacklist (UCSC Genome Browser) were removed. Exact sites of Tn5 transposase insertion were determined as the start position +5 bp for positive strand reads, and the end position -5 bp for negative strand reads \citep{Buenrostro2013}. Peaks were called using F-seq with default parameters. A union set of the top 50,000 peaks (ranked by F-seq score) from each sample was derived. Peaks were divided into overlapping 300 bp windows as previously described \citep{Shibata2012}. Per sample read coverage of each window was calculated using coverageBed from BedTools \citep{Quinlan2010}.

\subsection{Sequence trait filtering for QTL analysis}

Trimmed mean of M-values (TMM) normalization [edgeR, \citep{edgeR}] was applied to TPM values from read counts of genes and chromatin windows respectively. Genes with TMM-normalized TPM values $\leq$ 1 and chromatin windows with normalized counts $\leq$ 5 for $\geq$ 50\% of samples were excluded. For each gene and chromatin window, we applied $K$-means clustering with $K=2$ to identify outcomes containing outlier observations that could cause spurious, outlier-driven QTL calls. Any gene or chromatin window where the smaller $K$-means cluster had a cardinality of 1 was removed.

\subsection{Founder mosaic reduction}

CC genomes are mosaics of the founder strain haplotypes and were previously reconstructed by the UNC Systems Genetics Core (\url{http://csbio.unc.edu/CCstatus/index.py?run=FounderProbs}) with the Hidden Markov Model of \cite{Fu2012} on genotype calls [MegaMUGA array \citep{Morgan2016muga}] from multiple animals per strain. To reduce the number of statistical tests, adjacent genomic regions were merged through averaging if the founder mosaics for all mice were similar, defined as L2 distance $\leq$ 10\% of the maximum L2 distance ($\sqrt{2}$ for a probability vector).
