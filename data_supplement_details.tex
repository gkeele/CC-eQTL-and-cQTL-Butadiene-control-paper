\documentclass[10pt,letterpaper,twoside]{article}

\usepackage{gensymb}
\usepackage[labelformat=simple]{subcaption}
\renewcommand*{\thesubfigure}{\Alph{subfigure}}
\usepackage{tabularx, multirow}
\usepackage{makecell}
\usepackage{hyperref}
\hypersetup{draft}

\renewcommand{\figurename}{Fig}

\usepackage{siunitx}
\sisetup{output-exponent-marker=\ensuremath{\mathrm{e}}}

\bibliographystyle{plos2015}

\usepackage{threeparttable}
\usepackage[dvipsnames]{xcolor}
%\usepackage{enumitem}
%\newcommand{\method}[1]{\qquad Method #1}
\usepackage{graphicx}
\usepackage{soul}
\usepackage{xspace}
\usepackage[margin=1in]{geometry}
\newcommand{\indep}{\rotatebox[origin=c]{90}{$\models$}}
\newcommand{\eg}{\emph{e.g.}\xspace}
\newcommand{\ie}{\emph{i.e.}\xspace}
\newcommand{\T}{^\mathrm{T}}
\newcommand{\bbeta}{\boldsymbol{\beta}}
\newcommand{\blup}{\widetilde{\bbeta}}
\newcommand{\bzero}{\mathbf{0}}
\newcommand{\bI}{\mathbf{I}}
\newcommand{\bx}{\mathbf{x}}
\newcommand{\bX}{\mathbf{X}}
\newcommand{\by}{\mathbf{y}}
\newcommand{\tausq}{\tau^{2}}
\newcommand{\qtl}{\widehat{h^{2}}_{\text{QTL}}}

%\newlength{\myFootnoteLabel}
%\newenvironment{tableminipage}[1]{\begin{minipage}{#1}\renewcommand\footnoterule{ \kern -1ex}%
%\setlength{\myFootnoteLabel}{0.5em}%
%}{\end{minipage}}

\newcommand{\permpc}{\text{permP}_{\text{C}}}
\newcommand{\permpg}{\text{permP}_{\text{G}}}

\newcommand{\permpmed}{\text{permP}^{m}}

\newcommand{\ctwohtwo}{\texorpdfstring{C\textsubscript{2}H\textsubscript{2}}}

\begin{document}

\section*{Data and Supplement details}

Files  in the \textbf{Supplement} include:

\begin{itemize}
	\item File\_S1: \texttt{data\_supplement\_details.pdf}
	\item Results files:
	\begin{itemize}
		\item File\_S2 - S4: \texttt{de\_}\{\texttt{liver\_lung},\texttt{liver\_kidney},\texttt{lung\_kidney}\}\texttt{.csv}
		\item File\_S5 - S7: \texttt{dar\_}\{\texttt{liver\_lung},\texttt{liver\_kidney},\texttt{lung\_kidney}\}\texttt{.csv}
		\item File\_S8 - S10: \{\texttt{liver},\texttt{lung},\texttt{kidney}\}\texttt{\_local\_eqtl.csv}
		\item File\_S11 - S13: \{\texttt{liver},\texttt{lung},\texttt{kidney}\}\texttt{\_distal\_eqtl.csv}
		\item File\_S14 - S16: \{\texttt{liver},\texttt{lung},\texttt{kidney}\}\texttt{\_local\_cqtl.csv}
		\item File\_S17 - S19: \{\texttt{liver},\texttt{lung},\texttt{kidney}\}\texttt{\_distal\_cqtl.csv}
		\item File\_S20 - S22: \{\texttt{liver},\texttt{lung},\texttt{kidney}\}\texttt{\_chromatin\_mediation.csv}
	\end{itemize}
	\item Additional resource files:
	\begin{itemize}
		\item File\_S23: \texttt{supplement\_tables\_figures.pdf}
		\item File\_S24: \texttt{appendices.pdf}
	\end{itemize}
		\item Software files:
	\begin{itemize}
		\item File\_S25: \texttt{miqtl\_1.1.2.tar.gz}
	\end{itemize}
	\item R code:
	\begin{itemize}
		\item File\_S26: \texttt{generate\_manuscript\_figures.R}
		\item File\_S27: \texttt{generate\_supplement\_tables\_figures.R}
		\item File\_S28: \texttt{plotting\_functions.R}
		\item File\_S29: \texttt{analysis\_functions.R}
		\item File\_S30: \texttt{make\_qtl\_tables\_functions.R}
		\item File\_S31: \texttt{miqtl\_code\_demonstration.R}
	\end{itemize}
\end{itemize}

\noindent The data necessary for these analyses and raw QTL results files have been stored at figshare (doi:10.6084/m9.figshare.9985514). These files include:

\begin{itemize}
	\item Data files:
	\begin{itemize}
		\item \texttt{cc\_genome\_cache\_full\_l2\_0.1.zip}
		\begin{itemize}
			\item Founder haplotype mosaics for the 47 strains based on Build37, derived from files at \url{http://csbio.unc.edu/CCstatus/index.py?run=FounderProbs}, used for haplotype-based QTL analysis. Haplotype intervals were thinned, averaging adjacent intervals for which the mean l2 norm $<$ 0.1 \cite{Keele2019}.
		\end{itemize}
		\item \{\texttt{liver},\texttt{lung},\texttt{kidney}\}\texttt{\_expression.csv.zip}
		\item \{\texttt{liver},\texttt{lung},\texttt{kidney}\}\texttt{\_chromatin.csv.zip}
		\item \texttt{refseq\_mm9\_tss.txt.zip}
		\begin{itemize}
			\item Gene annotation data for Build37, providing gene transcription start site information necessary for determining local/distal status for eQTL. 
		\end{itemize}
		\item \texttt{pik3c2g\_phylogeny.csv}
		\begin{itemize}
			\item Maps the founder haplotypes around \textit{Pik3c2g} to the three $\textit{Mus}$ sub-species lineages (\url{http://msub.csbio.unc.edu}) \cite{Wang2012}.
		\end{itemize}
		\item \texttt{isvdb\_var\_db.tar.gz}
		\begin{itemize}
			\item SNP dosages derived from the founder haplotype mosaics and variant-to-strain distribution pattern, using ISVdb (\url{http://isvdb.unc.edu}) \cite{Oreper2017}.
		\end{itemize}
		\item \texttt{qtl\_intervals\_for\_var\_association.csv}
		\begin{itemize}
			\item Defined regions around detected QTL used to select variants for association analysis.
		\end{itemize}
	\end{itemize}
		\item Raw results files:
	\begin{itemize}
		\item \texttt{raw\_}\{\texttt{liver},\texttt{lung},\texttt{kidney}\}\texttt{.eqtl\_methodG.csv}
		\item \texttt{raw\_}\{\texttt{liver},\texttt{lung},\texttt{kidney}\}\texttt{.eqtl\_methodL.csv}
		\item \texttt{raw\_}\{\texttt{liver},\texttt{lung},\texttt{kidney}\}\texttt{.eqtl\_methodC.csv}
		\item \texttt{raw\_}\{\texttt{liver},\texttt{lung},\texttt{kidney}\}\texttt{.cqtl\_methodG.csv}
		\item \texttt{raw\_}\{\texttt{liver},\texttt{lung},\texttt{kidney}\}\texttt{.cqtl\_methodL.csv}
		\item \texttt{raw\_}\{\texttt{liver},\texttt{lung},\texttt{kidney}\}\texttt{.cqtl\_methodC.csv}
	\end{itemize}
\end{itemize}

\section*{miQTL package}

The static version of the miQTL R package (1.1.1) used for this work is provided in the \textbf{Supplement} as File\_S2. The current version of miQTL is available here: \url{https://github.com/gkeele/miqtl}. miQTL can be installed using the command `R CMD INSTALL' at the terminal. The current version from GitHub can be conveniently installed using the devtools R package and the following command within R: `install\_github(``gkeele/miqtl'')'.

\section*{File types}

\begin{itemize}
	\item *.csv - Comma-separated value files representing the trait data used for differential and QTL analyses and their large results tables, which were too large to be included as formatted tables.
	\item *.R - R scripts used to run the statistical analyses and generate the figures.
	\item *.RData - The zipped directory contained in \texttt{cc\_genome\_cache\_full\_l2\_0.1.zip} is composed of *.RData files that are HAPPY formatted, which the miQTL R package is designed to use.
	\item *.txt - The zipped text file \texttt{refseq\_mm9\_tss.txt.zip} is tab-delimited.
\end{itemize}

\thispagestyle{empty}
\newpage
\setcounter{page}{1}

\newpage

\begin{thebibliography}{10}

\bibitem{Keele2019}
Keele GR, Crouse WL, Kelada SNP, Valdar W.
\newblock {Determinants of QTL Mapping Power in the Realized Collaborative
  Cross.}
\newblock G3 (Bethesda, Md). 2019;9(May):459966.
\newblock doi:{10.1534/g3.119.400194}.

\bibitem{Wang2012}
Wang JR, de Villena FPM, McMillan L.
\newblock {Comparative analysis and visualization of multiple collinear genomes.}
\newblock BMC Bioinformatics. 2012;13(S3):S13.
\newblock doi:{10.1186/1471-2105-13-S3-S13}.

\bibitem{Oreper2017}
Oreper D, Cai Y, Tarantino LM, de~Villena FPM, Valdar W.
\newblock {Inbred Strain Variant Database (ISVdb): A Repository for
  Probabilistically Informed Sequence Differences Among the Collaborative Cross
  Strains and Their Founders.}
\newblock G3 (Bethesda, Md). 2017;7(6):1623--1630.
\newblock doi:{10.1534/g3.117.041491}.

\end{thebibliography}

\end{document}


